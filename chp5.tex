\chapter{Design projects}
\label{ch5}
This chapter elucidates the innovative work undertaken in this thesis.
\section{Multiple classes population}
Classifying cortical neurons into distinct classes provides a fundamental framework for understanding neural diversity. 
The four prominent cell classes in the cortex are \acrshort{fs}, \acrshort{rs}, 
\acrshort{ib}, inspired by Connors and Gutnick's classification \cite{CONNORS}, 
and the LTS cells.\\

RS neurons, characterised by a spiny pyramidal-cell morphology, comprise the 
predominant cell class in the neocortex. These neurons can exhibit either 
excitatory or inhibitory characteristics, with the former being more abundant. 
Moreover, the internal dynamics of these two cell types differ, necessitating 
the use of distinct models to represent each accurately.\\
The classification of neurons as excitatory or inhibitory is based on the 
neurotransmitter synthesised and released at the synapse. Glutamatergic 
neurons release glutamate, exciting postsynaptic cells, while GABAergic 
neurons release GABA (\(\gamma\)-aminobutyric acid), inhibiting them. 
Maintaining a delicate balance between excitation and inhibition is crucial 
for the precise functioning and plasticity of the nervous system.\\
Another significant class is the \acrshort{fs} neurons, typically 
corresponding to aspiny inhibitory neurons.\\
The \acrshort{ib} neurons, generating bursts of action potentials in response to 
depolarising stimuli, represent a smaller percentage of cells in the primary 
sensory cortex.\\
\acrshort{lts} activity was observed in about 10 \% of intracellularly 
recorded cells in the 
cat association cortex in vivo \cite{DestexheLTS}. These \acrshort{lts} neurons 
resemble the classic \acrshort{rs} response and additionally produce bursts of 
action potentials. While the LTS cell shares similarities with the thalamic relay 
cells, key distinctions exist. Thalamic relay cells lack spike-frequency adaptation, 
eliminating the need for adaptation currents. Furthermore, thalamic relay cells 
generate more powerful bursts than cortical \acrshort{lts} neurons in both the 
cortex and thalamus \cite{Pospischil}.\\
Despite these differences, the \acrshort{lts} class can be utilised to model 
thalamic neurons. A chip featuring these four neuronal classes can be a robust 
model for a thalamocortical network and prove valuable in various experimental 
scenarios, as previously done in other studies, like \cite{Khoyratee}.\\

\subsection{Implementation}
As illustrated in figure \ref{fig:schematic2}, the principal distinction 
from the design presented in Chapter \ref{section:PQNimpl} lies in the 
$PQN\ unit$ module. It features multiple $PQN\ engines$ (Figure \ref{fig:Pdesign1}), each equipped 
with a corresponding number of \acrshort{ram} blocks.\\

\begin{figure}[hbt!]
    \begin{center}
        \begin{circuitikz}
            \ctikzset{multipoles/dipchip/width=2}
        
            \node [left] at (-1,2.25) {$UART_{R}$};
            \node [left] at (-1,1) {$CLK$};
        
            \node [below, font=\tiny] at (0,0.4) {$CLK\ generator$};
            \draw (0,1) node[dipchip, scale=0.5, num pins=10, hide numbers, 
                no topmark, external pins width=0](Clk){}; % , label={Clk generator}
            \node [right, font=\tiny] at (Clk.bpin 3) {$C_{in}$};
            \node [left, font=\tiny] at (Clk.bpin 8) {$C$};
            % \node [left] at (-0.5,2) {$UART_{RXD}$};
            \draw (Clk.bpin 3) -- (-1,1);
        
            \node [below, font=\tiny] at (2,-0.6) {$Trigger\ generator$};
            \draw (2,0) node[dipchip, scale=0.5, num pins=10, hide numbers, 
                no topmark, external pins width=0](T){}; %, label={Trigger generator}
            \node [right, font=\tiny] at (T.bpin 3) {C};
            \node [left, font=\tiny] at (T.bpin 8) {T};
        
            \node [above, font=\tiny] at (4,2.6) {$Receiver$};
            \draw (4,2) node[dipchip, scale=0.5, num pins=10, hide numbers, 
                no topmark, external pins width=0](R){}; %label={I Receiver}
            \node [right, font=\tiny] at (R.bpin 2) {D};
            \node [right, font=\tiny] at (R.bpin 3) {C};
            \node [right, font=\tiny] at (R.bpin 4) {T};
            \node [left, font=\tiny] at (R.bpin 7) {$I_{3}$};
            \node [left, font=\tiny] at (R.bpin 8) {$I_{2}$};
            \node [left, font=\tiny] at (R.bpin 9) {$I_{1}$};
            \node [left, font=\tiny] at (R.bpin 10) {$I_{0}$};
            \draw (R.bpin 2) -- (-1,2.25);
        
            \node [above, font=\tiny] at (6,2.8) {$PQN_{unit}$};
            \draw (6,2) node[dipchip, scale=0.5, num pins=12, hide numbers, 
                no topmark, external pins width=0](PQN){}; %, label={PQN unit}
            \node [right, font=\tiny] at (PQN.bpin 1) {$I_{0}$};
            \node [right, font=\tiny] at (PQN.bpin 2) {$I_{1}$};
            \node [right, font=\tiny] at (PQN.bpin 3) {$I_{2}$};
            \node [right, font=\tiny] at (PQN.bpin 4) {$I_{3}$};
            \node [right, font=\tiny] at (PQN.bpin 5) {C};
            \node [right, font=\tiny] at (PQN.bpin 6) {T};
            \node [left, font=\tiny] at (PQN.bpin 8) {$V_{3}$};
            \node [left, font=\tiny] at (PQN.bpin 9) {$V_{2}$};
            \node [left, font=\tiny] at (PQN.bpin 10) {$V_{1}$};
            \node [left, font=\tiny] at (PQN.bpin 11) {$V_{0}$};
        
            \node [below, font=\tiny] at (8.1,1.2) {$Transmitter$};
            \draw (8,2) node[dipchip, scale=0.5, num pins=12, hide numbers, 
                no topmark, external pins width=0](Tr){}; %label={V Transmitter}
            \node [right, font=\tiny] at (Tr.bpin 1) {$V_{0}$};
            \node [right, font=\tiny] at (Tr.bpin 2) {$V_{1}$};
            \node [right, font=\tiny] at (Tr.bpin 3) {$V_{2}$};
            \node [right, font=\tiny] at (Tr.bpin 4) {$V_{3}$};
            \node [right, font=\tiny] at (Tr.bpin 5) {C};
            \node [right, font=\tiny] at (Tr.bpin 6) {T};
            \node [left, font=\tiny] at (Tr.bpin 9) {Q};
            \draw (Tr.bpin 10) -- ++ (0.5,0) node[right] {$UART_{T}$};
        
            \draw (T.bpin 8) -- ++ (4.5,0) |- (Tr.bpin 6);
            \draw (T.bpin 8) -- ++ (2.5,0) |- (PQN.bpin 6);
            \draw (T.bpin 8) -- ++ (0.5,0) |- (R.bpin 4);
        
            \draw[color=green] (Clk.bpin 8) -- ++ (6.35,0) |- (Tr.bpin 5); %[color=green]
            \draw[color=green] (Clk.bpin 8) -- ++ (4.4,0) |- (PQN.bpin 5); %[color=green]
            \draw[color=green] (Clk.bpin 8) -- ++ (2.4,0) |- (R.bpin 3); %[color=green]
            \draw[color=green] (Clk.bpin 8) -- ++ (0.4,0) |- (T.bpin 3); %[color=green]
        
            \draw (R.bpin 10) -- ++ (0.2,0) |- (PQN.bpin 1);
            \draw (R.bpin 9) -- ++ (0.2,0) |- (PQN.bpin 2);
            \draw (R.bpin 8) -- ++ (0.2,0) |- (PQN.bpin 3);
            \draw (R.bpin 7) -- ++ (0.2,0) |- (PQN.bpin 4);

            \draw (PQN.bpin 8) -- ++ (0.2,0) |- (Tr.bpin 4);
            \draw (PQN.bpin 9) -- ++ (0.15,0) |- (Tr.bpin 3);
            \draw (PQN.bpin 10) -- ++ (0.1,0) |- (Tr.bpin 2);
            \draw (PQN.bpin 11) -- ++ (0.05,0) |- (Tr.bpin 1);
        
        \end{circuitikz}
    \end{center}
    \caption[Multi-class population schematic diagram]{The schematic diagram for the implementation of a multi class population differs from \ref{fig:schematic} only for the number of currents output of the $receiver$ and thus in the number of voltages calculated by the $PQN\ unit$ and sent to the $transmitter$.}
    \label{fig:schematic2}
\end{figure}

\begin{figure}[hbt!]
    \begin{center}
    \begin{circuitikz}
        \ctikzset{multipoles/dipchip/width=2}
    
        \node[left] at (2.1,1) {$RS_{exci}\ engine$};
        \draw (0,0) node[dipchip, scale=0.3, num pins=10, hide numbers, 
            no topmark, external pins width=0](P1){}; % , label={Clk generator}
        \node [right, font=\tiny] at (P1.bpin 1) {C};
        \node [right, font=\tiny] at (P1.bpin 2) {$I_{rs}$};
        \node [right, font=\tiny] at (P1.bpin 3) {$v$};
        \node [right, font=\tiny] at (P1.bpin 4) {$n$};
        \node [right, font=\tiny] at (P1.bpin 5) {$q$};
        \node [left, font=\tiny] at (P1.bpin 8) {$v$};
        \node [left, font=\tiny] at (P1.bpin 7) {$n$};
        \node [left, font=\tiny] at (P1.bpin 6) {$q$};
    
        \draw (2,0) node[dipchip, scale=0.3, num pins=10, hide numbers, 
            no topmark, external pins width=0](V1){}; %, label={Trigger generator}
        \node [right, font=\tiny] at (V1.bpin 2) {C};
        \node [right, font=\tiny] at (V1.bpin 3) {D};
        \node [right, font=\tiny] at (V1.bpin 4) {$wr$};
        \node [right, font=\tiny] at (V1.bpin 5) {$adr$};
        \node [left, font=\tiny] at (V1.bpin 8) {Q};
        \draw (P1.bpin 8) -- (V1.bpin 3);
    
        \draw (2,-1) node[dipchip, scale=0.3, num pins=10, hide numbers, 
            no topmark, external pins width=0](N1){}; %, label={Trigger generator}
        \node [right, font=\tiny] at (N1.bpin 2) {C};
        \node [right, font=\tiny] at (N1.bpin 3) {D};
        \node [right, font=\tiny] at (N1.bpin 4) {$wr$};
        \node [right, font=\tiny] at (N1.bpin 5) {$adr$};
        \node [left, font=\tiny] at (N1.bpin 8) {Q};
        \draw (P1.bpin 7) -- ++ (0.7,0) |- (N1.bpin 3);
    
        \draw (2,-2) node[dipchip, scale=0.3, num pins=10, hide numbers, 
            no topmark, external pins width=0](Q1){}; %, label={Trigger generator}
        \node [right, font=\tiny] at (Q1.bpin 2) {C};
        \node [right, font=\tiny] at (Q1.bpin 3) {D};
        \node [right, font=\tiny] at (Q1.bpin 4) {$wr$};
        \node [right, font=\tiny] at (Q1.bpin 5) {$adr$};
        \node [left, font=\tiny] at (Q1.bpin 8) {Q};
        \draw (P1.bpin 6) -- ++ (0.5,0) |- (Q1.bpin 3);
    
        \draw[color=red] (Q1.bpin 4) -- ++ (-0.1,0);
        \draw[color=blue] (Q1.bpin 5) -- ++ (-0.1,0);
        \draw[color=red] (N1.bpin 4) -- ++ (-0.1,0);
        \draw[color=blue] (N1.bpin 5) -- ++ (-0.1,0);
        \draw[color=red] (V1.bpin 4) -- ++ (-0.1,0);
        \draw[color=blue] (V1.bpin 5) -- ++ (-0.1,0);
        \draw[color=green] (P1.bpin 1) -- ++ (-0.1,0);
        \draw[color=green] (V1.bpin 2) -- ++ (-0.1,0);
        \draw[color=green] (N1.bpin 2) -- ++ (-0.1,0);
        \draw[color=green] (Q1.bpin 2) -- ++ (-0.1,0);
    
        \draw (Q1.bpin 8) -- ++ (0.6,0) -- ++ (0,4) -- ++ (-4.5,0) -- ++ (0,-2.35) -- (P1.bpin 5);
        \draw (N1.bpin 8) -- ++ (0.4,0) -- ++ (0,2.7) -- ++ (-4,0) -- ++ (0,-1.87) -- (P1.bpin 4);
        \draw (V1.bpin 8) -- ++ (0.2,0) -- ++ (0,1.5) -- ++ (-3.5,0) -- ++ (0,-1.5) -- (P1.bpin 3);
    
        \node[left] at (7,1) {$FS\ engine$};
        \draw (5.5,0) node[dipchip, scale=0.3, num pins=10, hide numbers, 
            no topmark, external pins width=0](P2){}; % , label={Clk generator}
        \node [right, font=\tiny] at (P2.bpin 1) {C};
        \node [right, font=\tiny] at (P2.bpin 2) {$I_{fs}$};
        \node [right, font=\tiny] at (P2.bpin 3) {$v$};
        \node [right, font=\tiny] at (P2.bpin 4) {$n$};
        \node [right, font=\tiny] at (P2.bpin 5) {$q$};
        \node [left, font=\tiny] at (P2.bpin 8) {$v$};
        \node [left, font=\tiny] at (P2.bpin 7) {$n$};
        \node [left, font=\tiny] at (P2.bpin 6) {$q$};
    
        \draw (7.5,0) node[dipchip, scale=0.3, num pins=10, hide numbers, 
            no topmark, external pins width=0](V2){}; %, label={Trigger generator}
        \node [right, font=\tiny] at (V2.bpin 2) {C};
        \node [right, font=\tiny] at (V2.bpin 3) {D};
        \node [right, font=\tiny] at (V2.bpin 4) {$wr$};
        \node [right, font=\tiny] at (V2.bpin 5) {$adr$};
        \node [left, font=\tiny] at (V2.bpin 8) {Q};
        \draw (P2.bpin 8) -- (V2.bpin 3);
    
        \draw (7.5,-1) node[dipchip, scale=0.3, num pins=10, hide numbers, 
            no topmark, external pins width=0](N2){}; %, label={Trigger generator}
        \node [right, font=\tiny] at (N2.bpin 2) {C};
        \node [right, font=\tiny] at (N2.bpin 3) {D};
        \node [right, font=\tiny] at (N2.bpin 4) {$wr$};
        \node [right, font=\tiny] at (N2.bpin 5) {$adr$};
        \node [left, font=\tiny] at (N2.bpin 8) {Q};
        \draw (P2.bpin 7) -- ++ (0.7,0) |- (N2.bpin 3);
    
        \draw (7.5,-2) node[dipchip, scale=0.3, num pins=10, hide numbers, 
            no topmark, external pins width=0](Q2){}; %, label={Trigger generator}
        \node [right, font=\tiny] at (Q2.bpin 2) {C};
        \node [right, font=\tiny] at (Q2.bpin 3) {D};
        \node [right, font=\tiny] at (Q2.bpin 4) {$wr$};
        \node [right, font=\tiny] at (Q2.bpin 5) {$adr$};
        \node [left, font=\tiny] at (Q2.bpin 8) {Q};
        \draw (P2.bpin 6) -- ++ (0.5,0) |- (Q2.bpin 3);
    
        \draw[color=red] (Q2.bpin 4) -- ++ (-0.1,0);
        \draw[color=blue] (Q2.bpin 5) -- ++ (-0.1,0);
        \draw[color=red] (N2.bpin 4) -- ++ (-0.1,0);
        \draw[color=blue] (N2.bpin 5) -- ++ (-0.1,0);
        \draw[color=red] (V2.bpin 4) -- ++ (-0.1,0);
        \draw[color=blue] (V2.bpin 5) -- ++ (-0.1,0);
        \draw[color=green] (P2.bpin 1) -- ++ (-0.1,0);
        \draw[color=green] (V2.bpin 2) -- ++ (-0.1,0);
        \draw[color=green] (N2.bpin 2) -- ++ (-0.1,0);
        \draw[color=green] (Q2.bpin 2) -- ++ (-0.1,0);
    
        \draw (Q2.bpin 8) -- ++ (0.6,0) -- ++ (0,4) -- ++ (-4.5,0) -- ++ (0,-2.35) -- (P2.bpin 5);
        \draw (N2.bpin 8) -- ++ (0.4,0) -- ++ (0,2.7) -- ++ (-4,0) -- ++ (0,-1.87) -- (P2.bpin 4);
        \draw (V2.bpin 8) -- ++ (0.2,0) -- ++ (0,1.5) -- ++ (-3.5,0) -- ++ (0,-1.5) -- (P2.bpin 3);
    
        \node[left] at (1.5,-4) {$IB\ engine$};
        \draw (0,-5) node[dipchip, scale=0.3, num pins=12, hide numbers, 
            no topmark, external pins width=0](P3){}; % , label={Clk generator}
        \node [right, font=\tiny] at (P3.bpin 1) {C};
        \node [right, font=\tiny] at (P3.bpin 2) {$I_{ib}$};
        \node [right, font=\tiny] at (P3.bpin 3) {$v$};
        \node [right, font=\tiny] at (P3.bpin 4) {$n$};
        \node [right, font=\tiny] at (P3.bpin 5) {$q$};
        \node [right, font=\tiny] at (P3.bpin 6) {$u$};
        \node [left, font=\tiny] at (P3.bpin 10) {$v$};
        \node [left, font=\tiny] at (P3.bpin 9) {$n$};
        \node [left, font=\tiny] at (P3.bpin 8) {$q$};
        \node [left, font=\tiny] at (P3.bpin 7) {$u$};
    
        \draw (2,-4.9) node[dipchip, scale=0.3, num pins=10, hide numbers, 
            no topmark, external pins width=0](V3){}; %, label={Trigger generator}
        \node [right, font=\tiny] at (V3.bpin 2) {C};
        \node [right, font=\tiny] at (V3.bpin 3) {D};
        \node [right, font=\tiny] at (V3.bpin 4) {$wr$};
        \node [right, font=\tiny] at (V3.bpin 5) {$adr$};
        \node [left, font=\tiny] at (V3.bpin 8) {Q};
        \draw (P3.bpin 10) -- (V3.bpin 3);
    
        \draw (2,-5.9) node[dipchip, scale=0.3, num pins=10, hide numbers, 
            no topmark, external pins width=0](N3){}; %, label={Trigger generator}
        \node [right, font=\tiny] at (N3.bpin 2) {C};
        \node [right, font=\tiny] at (N3.bpin 3) {D};
        \node [right, font=\tiny] at (N3.bpin 4) {$wr$};
        \node [right, font=\tiny] at (N3.bpin 5) {$adr$};
        \node [left, font=\tiny] at (N3.bpin 8) {Q};
        \draw (P3.bpin 9) -- ++ (0.7,0) |- (N3.bpin 3);
    
        \draw (2,-6.9) node[dipchip, scale=0.3, num pins=10, hide numbers, 
            no topmark, external pins width=0](Q3){}; %, label={Trigger generator}
        \node [right, font=\tiny] at (Q3.bpin 2) {C};
        \node [right, font=\tiny] at (Q3.bpin 3) {D};
        \node [right, font=\tiny] at (Q3.bpin 4) {$wr$};
        \node [right, font=\tiny] at (Q3.bpin 5) {$adr$};
        \node [left, font=\tiny] at (Q3.bpin 8) {Q};
        \draw (P3.bpin 8) -- ++ (0.5,0) |- (Q3.bpin 3);

        \draw (2,-7.9) node[dipchip, scale=0.3, num pins=10, hide numbers, 
            no topmark, external pins width=0](U3){}; %, label={Trigger generator}
        \node [right, font=\tiny] at (U3.bpin 2) {C};
        \node [right, font=\tiny] at (U3.bpin 3) {D};
        \node [right, font=\tiny] at (U3.bpin 4) {$wr$};
        \node [right, font=\tiny] at (U3.bpin 5) {$adr$};
        \node [left, font=\tiny] at (U3.bpin 8) {Q};
        \draw (P3.bpin 7) -- ++ (0.3,0) |- (U3.bpin 3);
    
        \draw[color=red] (Q3.bpin 4) -- ++ (-0.1,0);
        \draw[color=blue] (Q3.bpin 5) -- ++ (-0.1,0);
        \draw[color=red] (N3.bpin 4) -- ++ (-0.1,0);
        \draw[color=blue] (N3.bpin 5) -- ++ (-0.1,0);
        \draw[color=red] (V3.bpin 4) -- ++ (-0.1,0);
        \draw[color=blue] (V3.bpin 5) -- ++ (-0.1,0);
        \draw[color=red] (U3.bpin 4) -- ++ (-0.1,0);
        \draw[color=blue] (U3.bpin 5) -- ++ (-0.1,0);

        \draw[color=green] (P3.bpin 1) -- ++ (-0.1,0);
        \draw[color=green] (V3.bpin 2) -- ++ (-0.1,0);
        \draw[color=green] (N3.bpin 2) -- ++ (-0.1,0);
        \draw[color=green] (Q3.bpin 2) -- ++ (-0.1,0);
        \draw[color=green] (U3.bpin 2) -- ++ (-0.1,0);
    
        \draw (U3.bpin 8) -- ++ (0.8,0) -- ++ (0,5.2) -- ++ (-5,0) -- ++ (0,-2.7) -- (P3.bpin 6);
        \draw (Q3.bpin 8) -- ++ (0.6,0) -- ++ (0,4) -- ++ (-4.5,0) -- ++ (0,-2.35) -- (P3.bpin 5);
        \draw (N3.bpin 8) -- ++ (0.4,0) -- ++ (0,2.7) -- ++ (-4,0) -- ++ (0,-1.89) -- (P3.bpin 4);
        \draw (V3.bpin 8) -- ++ (0.2,0) -- ++ (0,1.5) -- ++ (-3.5,0) -- ++ (0,-1.52) -- (P3.bpin 3);

        \node[left] at (7.2,-4) {$LTS\ engine$};
        \draw (5.5,-5) node[dipchip, scale=0.3, num pins=12, hide numbers, 
            no topmark, external pins width=0](P4){}; % , label={Clk generator}
        \node [right, font=\tiny] at (P4.bpin 1) {C};
        \node [right, font=\tiny] at (P4.bpin 2) {$I_{lts}$};
        \node [right, font=\tiny] at (P4.bpin 3) {$v$};
        \node [right, font=\tiny] at (P4.bpin 4) {$n$};
        \node [right, font=\tiny] at (P4.bpin 5) {$q$};
        \node [right, font=\tiny] at (P4.bpin 6) {$u$};
        \node [left, font=\tiny] at (P4.bpin 10) {$v$};
        \node [left, font=\tiny] at (P4.bpin 9) {$n$};
        \node [left, font=\tiny] at (P4.bpin 8) {$q$};
        \node [left, font=\tiny] at (P4.bpin 7) {$u$};
    
        \draw (7.5,-4.9) node[dipchip, scale=0.3, num pins=10, hide numbers, 
            no topmark, external pins width=0](V4){}; %, label={Trigger generator}
        \node [right, font=\tiny] at (V4.bpin 2) {C};
        \node [right, font=\tiny] at (V4.bpin 3) {D};
        \node [right, font=\tiny] at (V4.bpin 4) {$wr$};
        \node [right, font=\tiny] at (V4.bpin 5) {$adr$};
        \node [left, font=\tiny] at (V4.bpin 8) {Q};
        \draw (P4.bpin 10) -- (V4.bpin 3);
    
        \draw (7.5,-5.9) node[dipchip, scale=0.3, num pins=10, hide numbers, 
            no topmark, external pins width=0](N4){}; %, label={Trigger generator}
        \node [right, font=\tiny] at (N4.bpin 2) {C};
        \node [right, font=\tiny] at (N4.bpin 3) {D};
        \node [right, font=\tiny] at (N4.bpin 4) {$wr$};
        \node [right, font=\tiny] at (N4.bpin 5) {$adr$};
        \node [left, font=\tiny] at (N4.bpin 8) {Q};
        \draw (P4.bpin 9) -- ++ (0.7,0) |- (N4.bpin 3);
    
        \draw (7.5,-6.9) node[dipchip, scale=0.3, num pins=10, hide numbers, 
            no topmark, external pins width=0](Q4){}; %, label={Trigger generator}
        \node [right, font=\tiny] at (Q4.bpin 2) {C};
        \node [right, font=\tiny] at (Q4.bpin 3) {D};
        \node [right, font=\tiny] at (Q4.bpin 4) {$wr$};
        \node [right, font=\tiny] at (Q4.bpin 5) {$adr$};
        \node [left, font=\tiny] at (Q4.bpin 8) {Q};
        \draw (P4.bpin 8) -- ++ (0.5,0) |- (Q4.bpin 3);

        \draw (7.5,-7.9) node[dipchip, scale=0.3, num pins=10, hide numbers, 
            no topmark, external pins width=0](U4){}; %, label={Trigger generator}
        \node [right, font=\tiny] at (U4.bpin 2) {C};
        \node [right, font=\tiny] at (U4.bpin 3) {D};
        \node [right, font=\tiny] at (U4.bpin 4) {$wr$};
        \node [right, font=\tiny] at (U4.bpin 5) {$adr$};
        \node [left, font=\tiny] at (U4.bpin 8) {Q};
        \draw (P4.bpin 7) -- ++ (0.3,0) |- (U4.bpin 3);
    
        \draw[color=red] (Q4.bpin 4) -- ++ (-0.1,0);
        \draw[color=blue] (Q4.bpin 5) -- ++ (-0.1,0);
        \draw[color=red] (N4.bpin 4) -- ++ (-0.1,0);
        \draw[color=blue] (N4.bpin 5) -- ++ (-0.1,0);
        \draw[color=red] (V4.bpin 4) -- ++ (-0.1,0);
        \draw[color=blue] (V4.bpin 5) -- ++ (-0.1,0);
        \draw[color=red] (U4.bpin 4) -- ++ (-0.1,0);
        \draw[color=blue] (U4.bpin 5) -- ++ (-0.1,0);

        \draw[color=green] (P4.bpin 1) -- ++ (-0.1,0);
        \draw[color=green] (V4.bpin 2) -- ++ (-0.1,0);
        \draw[color=green] (N4.bpin 2) -- ++ (-0.1,0);
        \draw[color=green] (Q4.bpin 2) -- ++ (-0.1,0);
        \draw[color=green] (U4.bpin 2) -- ++ (-0.1,0);
    
        \draw (U4.bpin 8) -- ++ (0.8,0) -- ++ (0,5.2) -- ++ (-5,0) -- ++ (0,-2.7) -- (P4.bpin 6);
        \draw (Q4.bpin 8) -- ++ (0.6,0) -- ++ (0,4) -- ++ (-4.5,0) -- ++ (0,-2.35) -- (P4.bpin 5);
        \draw (N4.bpin 8) -- ++ (0.4,0) -- ++ (0,2.7) -- ++ (-4,0) -- ++ (0,-1.89) -- (P4.bpin 4);
        \draw (V4.bpin 8) -- ++ (0.2,0) -- ++ (0,1.5) -- ++ (-3.5,0) -- ++ (0,-1.52) -- (P4.bpin 3);

    \end{circuitikz}
    \end{center}
    \caption[Multiple classes population $PQN\ unit$ schematic diagram]{The \acrshort{pqn} unit is composed of 
    four independent and parallel compartments like the one in figure \ref{fig:PQNunit}.
    \acrshort{ib} and \acrshort{lts} classes have four-variables, 
    so their implementation requires an additional \acrshort{ram}. 
    The $wr$ and $adr$ signals are equal for all the blocks.}
    \label{fig:Pdesign1}
    \end{figure}

Within the $PQN\ unit$, four distinct $PQN\ engines$ have been 
incorporated — RSexci, \acrshort{fs}, \acrshort{lts}, and \acrshort{ib}. 
The design needs three \acrshort{ram}s for the first two classes and 
four for the latter two classes. Each class is associated with a unique 
stimulus current, differing in amplitude. Consequently, each of the four 
different currents must be transmitted to the respective class.\\
This required modifications to $transmitter$ and $receiver$, with the first 
that transfer four voltages 
and the second that receives four currents and stores them in the four \acrshort{ram}s.
The baud rate has been adjusted from 1 Mbits/s to 2 Mbit/s because of the increase in 
the data rate.\\
Figure \ref{fig:pqn4} shows the simultaneous stimulation of the four 
different classes with diverse currents.\\

The code relative to this section can be found in 
the $multiple\_classes$\footnote{\url{https://github.com/gle0/PQN_thesis/tree/main/multiple_classes}} repository on GitHub.\\

\begin{figure}[ht]
    \begin{center}
    \includegraphics[width=0.8\linewidth]{img/pqn4.PNG}
    \end{center}
    \caption[Simulation of multiple classes population]{Result of the stimulation of RSexci ($V_{0}$), FS ($V_{1}$), IB ($V_{2}$) and LTS ($V_{3}$) classes with a step current of 0.09, 0.1, 0.7 and -0.2, respectively.}
    \label{fig:pqn4}
\end{figure}

\clearpage

\section{Cell-to-cell variability}

In constructing thalamocortical networks, including diverse classes is essential, 
along with modelling the inherent cell-to-cell variability within each 
class \cite{Pospischil}. Unlike computers with billions of identical gates, 
neurons in the brain exhibit significant individuality \cite{Stein,Urban}. 
Even neurons of the same type display considerable variances in their properties, 
introducing variability to signal processing. While this variability might seem to 
challenge the reliability of neuronal networks, it has been observed that instead it 
enhances a system's inherent functional behaviour \cite{Balasubramanian,Perez-Nievez,Zendrikov}. 
In fact, this intrinsic mechanism contributes to increased network speed, 
responsiveness, and robustness \cite{Lengler}.\\

\subsection{Implementation}

To exploit the cell-to-cell variability, it’s pivotal to provide each 
neuron with a heterogeneous set of parameters (seen the PQN model's 
equations in Chapter \ref{subsection:PQN}). 
In this project the coefficients discussed in Chapter \ref{section:PQNimpl} 
are a semplification of those parameters.\\
In order for each neuron to have a proper set of coefficients, it’s important 
to make them change at any cycle rather than being constants. In fact, the 
coefficients in \cite{Nanami} were fixed, allowing for straightforward 
multiplication using simple shifters and adders. However, this approach is no 
longer applicable, and a novel solution is required. Instead of constructing 
hardware around fixed values for the coefficient, the goal is to develop generic 
hardware capable of accommodating any value.\\

Therefore, there is the need for a $barrel\ shifter$ module 
(Figure \ref{fig:barrs}) capable of shifting a variable based on a number. 
This module takes a variable and a 5-bit signal ($shift\ amount$) as input, 
outputting the variable shifted by the specified number \cite{SchultePillmeier}. 
The shift amount determines the positions by which the input is shifted.\\

\begin{figure}[hbt!]
    \begin{center}
        \begin{circuitikz}
            \draw (0,0) rectangle (2,2.3);
            \draw (0,1.5) -- (-1,1.5) node[left] {$IN$};
            \draw (0,0.8) -- (-1,0.8) node[left] {$SHIFT_{AMT}$};
            \draw (2,1.15) -- ++ (0.5,0) node[right] {$OUT$};
            \node[scale=2] at (1,1.15) {$BS$};
        \end{circuitikz}
    \end{center}
    \caption[Barrel shifter module]{The barrel shifter is a module composed of many \acrshort{mux}s. 
    It takes two inputs, one is a variable and the other one serves as the selector for these \acrshort{mux}s. 
    This module allows to shift a variable by any value.}
    \label{fig:barrs}
\end{figure}

Another challenge arises in determining how to store these dynamic coefficients. 
Since all the values are binary, only the 1 bits are relevant to the multiplication 
operation. To address this, a strategy is employed to store the positions of 1s 
exclusively.\\
For instance, the binary coefficient 0000100110 (10-bit fixed-point, with a 1-bit 
integer part, equivalent to 0.07421875 in decimal) is encoded by the 
positions 4, 7, and 8 expressed in a 5-bit notation (00100, 00111, 01000).\\
The maximum number of 1s that can be stored for each coefficient has been 
determined as 10 considering the coefficients in \cite{Nanami}. Summing up, 
ten positional values are stored for each coefficient, with each position 
expressed using 5 bits (Figure \ref{fig:50bits} top). After doing this, 
the ten shifted values are summed, giving the result of the multiplication 
between a variable and a coefficient.\\
Consequently, for each neuron 17 different 50-bit arrays are stored in the 
respective \acrshort{rom} block. In cases with fewer than ten 1s, the remaining operations 
involve a 37-bit shifting, resulting in an empty array. The addition of these null 
values does not impact the calculation of the coefficient (Figure \ref{fig:50bits} 
bottom).\\

\begin{figure}[hbt!]
    \begin{center}
    \includegraphics[width=1\linewidth]{img/50bitsss.png}
    \end{center}
    \caption[Encoding a coefficient in a 50-bits array]{In the first case (top) the coefficient is encoded using 10 1s, in the second (bottom) case less than 10.}
    \label{fig:50bits}
\end{figure}

The overall design is illustrated in figure \ref{fig:schematic3}. 
The modification is focused on the $PQN\ engine$, which incorporates 
17 \acrshort{rom}s (one for each coefficient) and 10 $barrel\ shifters$ for each \acrshort{rom}.\\

\begin{figure}[hbt!]
    \begin{center}
        \begin{circuitikz}
            \ctikzset{multipoles/dipchip/width=2}
        
            \node [left] at (-1,2.25) {$UART_{R}$};
            \node [left] at (-1,0.454) {$CLK$};
        
            \node [below, font=\tiny] at (0,-0.15) {$CLK\ generator$};
            \draw (0,0.454) node[dipchip, scale=0.5, num pins=10, hide numbers, 
                no topmark, external pins width=0](Clk){}; % , label={Clk generator}
            \node [right, font=\tiny] at (Clk.bpin 3) {$C_{in}$};
            \node [left, font=\tiny] at (Clk.bpin 8) {$C$};
            \draw (Clk.bpin 3) -- (-1,0.454);
        
            \node [below, font=\tiny] at (2,-1.05) {$Trigger\ generator$};
            \draw (2,-0.5) node[dipchip, scale=0.5, num pins=10, hide numbers, 
                no topmark, external pins width=0](T){}; %, label={Trigger generator}
            \node [right, font=\tiny] at (T.bpin 3) {C};
            \node [left, font=\tiny] at (T.bpin 8) {T};
        
            \node [above, font=\tiny] at (4,2.6) {$Receiver$};
            \draw (4,2) node[dipchip, scale=0.5, num pins=10, hide numbers, 
                no topmark, external pins width=0](R){}; %label={I Receiver}
            \node [right, font=\tiny] at (R.bpin 2) {D};
            \node [right, font=\tiny] at (R.bpin 3) {C};
            \node [right, font=\tiny] at (R.bpin 4) {T};
            \node [left, font=\tiny] at (R.bpin 8) {Q};
            \draw (R.bpin 2) -- (-1,2.25);
        
            \node [above, font=\tiny] at (6,3.3) {$PQN_{unit}$};
            \draw (6,2) node[dipchip, scale=0.5, num pins=20, hide numbers, 
                no topmark, external pins width=0](PQN){}; %, label={PQN unit}
            \node [right, font=\tiny] at (PQN.bpin 5) {I};
            \node [right, font=\tiny] at (PQN.bpin 6) {C};
            \node [right, font=\tiny] at (PQN.bpin 7) {T};
            \node [left, font=\tiny] at (PQN.bpin 11) {$V_{9}$};
            \node [left, font=\tiny] at (PQN.bpin 12) {$V_{8}$};
            \node [left, font=\tiny] at (PQN.bpin 13) {$V_{7}$};
            \node [left, font=\tiny] at (PQN.bpin 14) {$V_{6}$};
            \node [left, font=\tiny] at (PQN.bpin 15) {$V_{5}$};
            \node [left, font=\tiny] at (PQN.bpin 16) {$V_{4}$};
            \node [left, font=\tiny] at (PQN.bpin 17) {$V_{3}$};
            \node [left, font=\tiny] at (PQN.bpin 18) {$V_{2}$};
            \node [left, font=\tiny] at (PQN.bpin 19) {$V_{1}$};
            \node [left, font=\tiny] at (PQN.bpin 20) {$V_{0}$};
        
            \node [below, font=\tiny] at (8.1,0.15) {$Transmitter$};
            \draw (8,1.72) node[dipchip, scale=0.5, num pins=24, hide numbers, 
                no topmark, external pins width=0](Tr){}; %label={V Transmitter}
            \node [right, font=\tiny] at (Tr.bpin 1) {$V_{0}$};
            \node [right, font=\tiny] at (Tr.bpin 2) {$V_{1}$};
            \node [right, font=\tiny] at (Tr.bpin 3) {$V_{2}$};
            \node [right, font=\tiny] at (Tr.bpin 4) {$V_{3}$};
            \node [right, font=\tiny] at (Tr.bpin 5) {$V_{4}$};
            \node [right, font=\tiny] at (Tr.bpin 6) {$V_{5}$};
            \node [right, font=\tiny] at (Tr.bpin 7) {$V_{6}$};
            \node [right, font=\tiny] at (Tr.bpin 8) {$V_{7}$};
            \node [right, font=\tiny] at (Tr.bpin 9) {$V_{8}$};
            \node [right, font=\tiny] at (Tr.bpin 10) {$V_{9}$};
            \node [right, font=\tiny] at (Tr.bpin 11) {C};
            \node [right, font=\tiny] at (Tr.bpin 12) {T};
            \node [left, font=\tiny] at (Tr.bpin 18) {Q};
            \draw (Tr.bpin 18) -- ++ (0.5,0) node[right] {$UART_{T}$};
        
            \draw (T.bpin 8) -- ++ (4.5,0) |- (Tr.bpin 12);
            \draw (T.bpin 8) -- ++ (2.5,0) |- (PQN.bpin 7);
            \draw (T.bpin 8) -- ++ (0.5,0) |- (R.bpin 4);
        
            \draw[color=green] (Clk.bpin 8) -- ++ (6.35,0) |- (Tr.bpin 11); %[color=green]
            \draw[color=green] (Clk.bpin 8) -- ++ (4.4,0) |- (PQN.bpin 6); %[color=green]
            \draw[color=green] (Clk.bpin 8) -- ++ (2.4,0) |- (R.bpin 3); %[color=green]
            \draw[color=green] (Clk.bpin 8) -- ++ (0.4,0) |- (T.bpin 3); %[color=green]
        
            \draw (R.bpin 8) -- ++ (0.2,0) |- (PQN.bpin 5);

            \draw (PQN.bpin 20) -- (Tr.bpin 1);
            \draw (PQN.bpin 19) -- (Tr.bpin 2);
            \draw (PQN.bpin 18) -- (Tr.bpin 3);
            \draw (PQN.bpin 17) -- (Tr.bpin 4);
            \draw (PQN.bpin 16) -- (Tr.bpin 5);
            \draw (PQN.bpin 15) -- (Tr.bpin 6);
            \draw (PQN.bpin 14) -- (Tr.bpin 7);
            \draw (PQN.bpin 13) -- (Tr.bpin 8);
            \draw (PQN.bpin 12) -- (Tr.bpin 9);
            \draw (PQN.bpin 11) -- (Tr.bpin 10);
        
        \end{circuitikz}
    \end{center}
    \caption[Heterogeneous population schematic diagram]{The schematic diagram for the heterogeneous population differs from the one in figure \ref{fig:schematic} only in the number of voltages computed by the $PQN\ unit$ and sent to the $transmitter$.}
    \label{fig:schematic3}
\end{figure}

The revised $PQN\ engine$ operates through a new process involving eight distinct stages (Figure \ref{fig:8stages}). 
The initial stage involves the computation of \(v^2\). In the second ten shifted 
values are calculated for each coefficient. In the third stage, these ten values 
are summed in pairs. This results in five values, summed together in the fourth 
and the fifth stages. In the sixth stage, nine coefficients become negative. The 
seventh stage computes \(v_{out}\), \(n_{out}\), and \(q_{out}\). Finally, in the 
eighth stage, these values are outputted.\\

\begin{figure}[hbt!]
    \begin{center}
    \includegraphics[width=0.84\linewidth]{img/block_diagram2.png}
    \end{center}
    \caption[$PQN\ engine$ pipeline for variable coefficients]{PQN engine diagram for the three variables model that implement neuronal heterogeneity. $BS$ is the symbol indicating the barrel shifter.}
    \label{fig:8stages}
\end{figure}

One current is received and ten voltages are transmitted. The baud rate has 
thus been increased from 2 Mbits/s to 4 Mbits/s.\\
Figure \ref{fig:ctc var} shows the stimulation of a heterogeneous population 
of RSexci neurons with the same step input current.\\

\begin{figure}[hbt!]
    \begin{center}
    \includegraphics[width=1\linewidth]{img/het_pop.PNG}
    \end{center}
    \caption[Stimulation of heterogeneous neurons]{Stimulation of 5 RSexci neurons with a step current. 
    Although the neurons are all from the same class, their response to the same current is different.}
    \label{fig:ctc var}
\end{figure}

The relevant code for this section can be found in the GitHub 
repository called $cell-to-cell\_variability$\footnote{\url{https://github.com/gle0/PQN_thesis/tree/main/cell-to-cell_variability}}.\\
