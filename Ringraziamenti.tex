\chapter*{Ringraziamenti}

Mi sembra doveroso fare una premessa prima di dedicare un pensiero alle persone grazie alle quali sono arrivato qui. Come la maggior parte di chi sta leggendo sa, esprimermi in modo chiaro non è mai stato il mio forte. Nonostante questo, ho cercato di fare del mio meglio per tirare delle somme quantomeno comprensibili. Oltre queste due pagine ci sono un’infinità di altri pensieri e sentimenti legati a voi che non riuscirò a imprimere su carta, ma che porto sempre con me.\\

Ci tengo a iniziare ringraziando la professoressa Michela Chiappalone, il professore Timothée Levi e il professore Takashi Kohno per aver reso possibile quello che un anno fa non mi sarei potuto nemmeno immaginare. L’opportunità di studiare quello che più mi interessava in una città speciale come Tokyo ha rappresentato un’esperienza unica, che mi ha permesso di realizzare davvero quanto sia vasto il mondo.
La mia sopravvivenza lì è stata indubbiamente merito di Izubuchi-san, che è sempre stata pronta a darmi una mano in qualsiasi momento, preoccupandosi per me come se fossi un membro della sua famiglia. Invece Nanami-san mi ha seguito e sostenuto per tutta la durata del progetto di tesi. Un grazie a loro due e a tutti i nuovi amici che ho conosciuto in quei sei mesi.\\

Guardando al passato durante i miei anni di università, mi accorgo che, malgrado gli sforzi per rimanere in carreggiata, in qualche modo ho sempre avuto la tendenza a deviare un po'. Sarò eternamente grato a coloro che, con piccoli e grandi gesti, mi hanno aiutato a rimettere a posto la rotta. Ai miei amici del PoliTo, che più che compagni di corso sono stati compagni di squadra, facendomi dono della cosa che più mi manca da quando ho lasciato Torino: il calcetto. Ai miei amici di Genova, coinquilini insieme a me del simpatico stabile chiamato edificio E, che hanno camminato accanto a me nel percorso magistrale. A Triso e a Pasca per essere stati il mio punto fermo in tutti questi anni.\\
Arianna è la mia eroina, la ringrazio dal profondo del cuore. Per essersi presa cura di me. Per non avermi lasciato andare. Per essermi stata accanto, evitando che questo ultimo anno mi sopraffacesse. Per essere la persona che ambisco a diventare. Se ripenso a dove sono arrivato, poi, un grazie va alla persona che da piccolo mi ha fatto amare la matematica, e che ci ha lasciati troppo presto.\\

Dedico un ringraziamento speciale a tutta la mia famiglia. A mamma e papà, che mi hanno sempre messo davanti a loro e hanno accettato qualsiasi scelta io prendessi, credendo ciecamente in me e nelle mie passioni. La libertà che mi hanno donato è inestimabile. Questo traguardo è merito loro. A mia sorella, che è quella uscita bene in famiglia, per riuscire a farmi ridere nei modi più strani. Per ricordarmi casa ovunque vada. Per volermi bene come nessun altro.\\
Infine, il ringraziamento per me più sentito è dedicato ai miei nonni, che mi sostengono da tutta la vita senza mai aver voluto niente in cambio, facendo sembrare il loro amore per me la cosa più naturale del mondo. L’ultimo grazie è rivolto al cielo, da dove in questo momento ci starà guardando nonno Raffaele, già pronto con in alto il calice.\\

Un brindisi a tutti voi,
\begin{flushright}
    \textit{Giuseppe}
\end{flushright}