\chapter{Computational Neuron Models}
\label{ch3}
In biology, the nervous system is the central 
processing and communication hub within an 
organism, facilitating the transmission of 
signals to and from various body parts to 
control, regulate, and coordinate its functions. 
It comprises two types of cells: neurons and 
glial cells. Glial cells play a crucial role in 
stabilisation, nutrition, and 
protection \cite{Jessen}.\\
On the other hand, neurons are highly 
specialised cells responsible for generating 
and transmitting electrical signals to other 
cells. They consist of a cell body, or 
soma, which gives rise to two distinct 
ramifications: dendrites and axons. The dendrites 
receive stimuli from other neurons and 
convey them towards the cell 
body. At the same time, the axons transport the 
neural signal from the cell body to the synapses,
sites specialised in 
transferring the signals to other 
cells \cite{Dayan}.\\

For transmitting information, neurons utilise 
\acrfull{ap} (Figure \ref{fig:spike}), 
which are rapid fluctuations in their membrane 
potential that propagate along 
the neuronal membrane. They are generated by 
specific types of 
voltage-gated ion channels.\\
When a neuron receives a 
synaptic input, the membrane depolarises, 
resulting in a rise in its potential. This 
permits an influx of sodium 
ions in the cell, which leads to a strong 
increase in 
the potential. This induces a reversal 
in the voltage polarity, followed by the closure 
of the sodium channels and the activation of the
potassium channels, initiating an outward flow 
of potassium ions. This outward current restores 
the electrochemical gradient, returning the 
membrane potential to its resting 
state \cite{Dayan}.\\

\begin{figure}[hbt!]
    \begin{center}
    \includegraphics[width=0.7\linewidth]{img/Diagram-for-action-potential.png}
    \end{center}
    \caption[Shape of the Action Potential]{Picture taken from \cite{Sharma}. 
    With the arriving of input current, there’s the accumulation of the inputs that increase the membrane potential. When it reaches a threshold, all the channels open, bringing to the actual depolarization, followed by the repolarisation. A spike happens. After that, there’s a hyperpolarisation, and the neuron enters a resting period, in which it can’t fire. The \acrshort{ap} shape can be modelled with an exponential function.}
    \label{fig:spike}
\end{figure}

From a computational perspective, neurons, 
the key computational elements in biological 
systems, perform nonlinear transformations on 
their inputs in both spatial and temporal 
dimensions. As illustrated in 
figure \ref{fig:neuron morphology}, 
dendrites function as the input stage, 
the soma operates the computational processes, 
and the resulting output
is transmitted along the axon. It, 
in turn, connects to the dendrites of other 
neurons through synapses. In many neuromorphic 
systems, the soma is commonly referred to 
simply as the entire neuron.\\

\begin{figure}[hbt!]
    \begin{center}
    \includegraphics[width=0.5\linewidth]{img/neu_morp_paper.png}
    \end{center}
    \caption[Neuron morphology]{Image taken from \cite{Walter}. Neurons can have diverse morphologies. Here, a pyramidal neuron, the most plentiful type of neuron in the cortex, is illustrated. While presenting morphological differences, all the neurons share the same functional structures. The dendrites are short extensions that originate in different parts of the cell. They receive input stimuli from other neurons’ synapses. These arrive at the soma, which is the cellular main body. Its form varies with the cell type and its dimensions range from 4-140 \textmu m. It processes the inputs and produces an output that travels along the axon, a long branch that has many synapses at its end.}
    \label{fig:neuron morphology}
\end{figure}

Models of neural computation try to elucidate, in an abstract and mathematical manner, the fundamental principles that govern information processing in biological nervous systems \cite{Koch2001,Koch2004}. Neuron models are mathematical descriptions of how the voltage potential across the neuron membrane changes over time, with the first such model dated 1907 \cite{Lapicque}.\\
Today, based on biological realism, neuron models can be broadly categorised into two classes: biophysical and abstracted neural models. Biophysical models aim for a detailed and realistic representation of the biological processes within neurons, while abstracted models simplify the representation to emphasise the essential aspects.\\
In the context of neuromorphic applications, the following lists the most used neuron models.\\

\section{Neuron models for neuromorphic applications}
\subsection{Integrate-and-Fire models}
In 1907, Louis Lapique introduced the \acrfull{if} model \cite{Abbott1999,Tuckwell}, which belongs to the category of abstracted models. This simple model represents neurons using an electric circuit consisting solely of a capacitor. The capacitor models the neuronal membrane. When it is charged to a specific threshold potential, an \acrshort{ap} is generated, with the subsequent capacitor discharge.\\
One of the most used \acrshort{if} models is the \acrfull{lif} model \cite{Burkitt}, which incorporates a leakage term (resistance) into the \acrshort{if} circuit (Figure \ref{fig:LIFcircuit}). This accounts for voltage-independent channels that are responsible for the leakage of ions into and out of the cell, giving rise to a net leakage current $I_{leak}$. From a modelling perspective, the introduction of this new term enables the incorporation of time-dependent memory into the system.\\

The equations of the LIF model are \ref{eq:LIF1} and \ref{eq:LIF2}\\
\begin{equation}
    C\frac{dV}{dt}=-g_{L}\left(v-E_{L}\right)+I(t)\
    \label{eq:LIF1}
    \end{equation}

\begin{equation}
    v=v_{reset}\ \ if\ v>v_{th}
    \label{eq:LIF2}
    \end{equation}

In \ref{eq:LIF1} \(I(t)\) is the synaptic or input current at time t,
\(C\) the membrane capacitance,
\(g_{L}\) the membrane conductance and
\(E_{L}\) the resting membrane potential.\\

\begin{figure}[hbt!]
    \centering
    \begin{circuitikz}
        % Neuron components
        \draw (0,0) node[circle, draw, inner sep=1pt]{};
        \draw (5,0) node[circle, draw, inner sep=1pt]{};
        \draw (5,0) to[R, l=$R$] (2,0) to[C, l=$C$] (2,-2) node[ground]{};
        % Voltage source and label
        \draw (2,0) -- (0,0) node[left]{I};
        \draw (5,0) node[right]{$V_{\text{rest}}$};
    \end{circuitikz}
    \caption[LIF model circuit]{The subthreshold regime of the model is described by the RC circuit dynamics. When a current is applied across the circuit, the capacitor starts to charge. 
    During the charging phase, the voltage across the capacitor ($V_{c}$) increases over time. 
    The rate at which the capacitor charges is determined by the time constant ($\tau=RC$) of the circuit.
    The charging of the capacitor follows an exponential function given by the formula $V_{c}(t)=V_{max}(1-e^{-t/\tau})$, 
    where $V_{max}$ is the maximum value that can reach the voltage\\
    When there is no input current, the capacitor starts to discharge through the resistor,
    following an exponential decay given by $V_{c}(t)=V_{0}e^{-t/\tau}$, where $V_{0}$ is the initial value of the voltage.}
    \label{fig:LIFcircuit}
\end{figure}

Figure \ref{fig:lif} explains the model working principle: neurons integrate incoming inputs from external signals or synapses and generate an output, firing an action potential. A key feature shared by the IF models is the reset of the state variable v when it reaches a predetermined threshold. When this happens, a spike is registered at time \(t_{0}\). In these models, the exact shape of the action potential is not crucial; what matters is the time of its occurrence. Consequently, the output of the simulation is a spike train, with \(t_{0}\) values corresponding the times at which action potentials are elicited.\\

\begin{figure}[hbt!]
    \begin{center}
    \includegraphics[width=0.8\linewidth]{img/LLLif1.png}
    \end{center}
    \caption[LIF model functioning]{On the top, there are the spikes in inputs to the model; the central plot shows the dynamics of the membrane potential that change according to the inputs; on the bottom, there are output spikes of the model, produced when the membrane potential exceeds the membrane threshold.}
    \label{fig:lif}
\end{figure}

The LIF model, while simple, has limitations in reproducing various neuronal dynamics, such as firing rate adaptation, and burst. To overcome these limitations, researchers have developed new IF models with enhanced capabilities. Some of these include the \acrfull{qif} \cite{Zheng}, the \acrfull{eif} \cite{FourcaudTrocme}, the \acrfull{ifb} \cite{Smith}, the \acrfull{adex} \cite{Brette}, and the \acrfull{izh} \cite{IZH}.\\

\subsection{Conductance-based models}
Conductance-based models \cite{Hendrickson} offer a straightforward biophysical representation of a neuron, where protein molecule ion channels are represented by conductances and the lipid bilayer by a capacitor. These models are grounded in an equivalent circuit representation of a cell membrane as well. This framework is characterised by a single isopotential electrical compartment, accounting for ion movements between the interior and exterior of the cell. The membrane potential $v$ is influenced by the flow of ions, (such as sodium, potassium, and calcium) through various ion channels across the neuronal membrane.\\
The \acrfull{hh} model, introduced in 1952 \cite{HH} and based on the giant squid axon, stands as the pioneering conductance-based model. In this instance, the included ion channels are those for sodium (Na) and potassium (K) (figure \ref{fig:hh}).\\
    
\begin{figure}[hbt!]
    \begin{center}
    \begin{circuitikz}
    \draw (3,-3) node[circle, fill=black, draw, inner sep=1pt]{};
    \draw (3,-3) -- (3,-2);
    \draw (3,3) node[circle, fill=black, draw, inner sep=1pt]{};
    \draw (3,3) -- (3,2);
    \draw (0,-2) -- (6,-2);
    \draw (0,2) -- (6,2);
    
    % Capacitor
    \draw (0,2) to[C, l=$C_m$] (0,-2);

    % Sodium branch
    \draw (2,0) to[R, l=$R_{\text{Na}}$] (2,2);
    \draw [->, rotate=350] (1.8,0.75) -- (1.9,1.75);
    \draw (2,-2) to[V, v=$E_{\text{Na}}$] (2,0);

    % Potassium branch
    \draw (4,0) to[R, l=$R_{\text{K}}$] (4,2);
    \draw [->, rotate=350] (3.8,1.1) -- (3.9,2.1);
    \draw (4,-2) to[V, v=$E_{\text{K}}$] (4,0);


    % Leak branch
    \draw (6,0) to[R, l=$R_{\text{Leak}}$] (6,2);
    \draw (6,0) to[V, v=$E_{\text{Leak}}$] (6,-2);
    \end{circuitikz}
    \end{center}
    \caption[Hodgkin-Huxley model circuit]{The \acrshort{hh} model accurately simulates the generation and propagation of action potentials in neurons. Its whole dynamics are modelled through an RC circuit. The most relevant parts of this circuit are $R_{Na}$ and $R_{K}$. These resistances vary over time depending on the voltage. This introduces two new currents to the model that represent the flow of the sodium and potassium ions.}
    \label{fig:hh}
\end{figure}
Therefore, the model incorporates a sodium current \(I_{Na}\) with activation (m) and inactivation (h) variables, a potassium current \(I_{K}\) with only an activation (n) variable. This allows to accurately capture the dynamics of action potential generation.\\
The model's behaviour is described with a system of differential equations, where each equation represents the dynamics of a specific variable, such as membrane potential or ion conductance's activation and inactivation variables.

\begin{equation}
    \begin{split}
    C\frac{dv}{dt}&=-I_{Na}-I_{K}-I_{leak}+I\\
    &=-{\overline{g}}_{Na}m^3h\left(v-E_{Na}\right)-{\overline{g}}_Kn^4\left(v-E_K\right)-g_L\left(v-E_L\right)+I
    \end{split}
    \label{eq:hh1}
\end{equation}

\begin{equation}
    \frac{dn}{dt}=\alpha_n\left(v\right)\left(1-n\right)-\beta_n\left(v\right)n
    \label{eq:hh2}
\end{equation}

\begin{equation}
    \frac{dm}{dt}=\alpha_m\left(v\right)\left(1-m\right)-\beta_m\left(v\right)m
    \label{eq:hh3}
\end{equation}

\begin{equation}
    \frac{dh}{dt}=\alpha_h\left(v\right)\left(1-h\right)-\beta_h\left(v\right)h
    \label{eq:hh4}
\end{equation}

In equations \ref{eq:hh1} \({\overline{g}}_{X}\) is the maximum value for the ionic conductances,
\(alpha\) and \(beta\) are voltage dependent parameters and
\(E_{X}\) the ion equilibrium potential. It is the membrane potential at which there is no net flow of ions.\\


\subsection{Qualitative models}
Qualitative models are simplified representations of conductance-based models, aiming to replicate neural dynamics using reduced equations. This simplification is achieved by focusing on the mathematical structures in ionic conductance models rather than delving into the details of ion charge dynamics.\\
The history of qualitative modelling traces back to the FitzHugh–Nagumo model \cite{FitzHugh,Nagumo} and finds a theoretical basis in the theory of nonlinear dynamics \cite{Kepler,Rinzel,Strogatz,Izhikevich2007}.\\

The Fitzhugh-Nagumo model is well-suited for describing the dynamics of excitable cells, this model captures only the essential features of action potential generation, using:\\

1.	A voltage-like variable $v$ with cubic nonlinearity that allows regenerative self-excitation via positive feedback.

\begin{equation}
    \frac{dv}{dt}=v-\frac{v^3}{3}-w+I
    \label{eq:fn1}
\end{equation}


2.	A recovery variable $w$ with linear dynamics, providing slower negative feedback and associated with the activation and inactivation of ion channels.

\begin{equation}
    \frac{dw}{dt}=\frac{1}{\tau}(v+\alpha-bw)
    \label{eq:fn2}
\end{equation}

In equations \ref{eq:fn1} and \ref{eq:fn2} \(\tau\) is time constant of the recovery variable,
\(a\), \(b\) are model constant parameters.

\section{Piecewise Quadratic Neuron model}
\subsection{Reason behind the PQN model}
A diverse range of neuron models has been employed in both large-scale \acrshort{snn} models and \acrshort{sinn}s \cite{Frenkel2023}. Within this context, achieving a balance between the reproducibility of neuronal activities and computational efficiency is crucial.\\

The nervous system comprises various neuronal classes, each characterised by distinct electrophysiological properties, known to play an essential role in information processing \cite{Cunningham,Benda,GRILLNER}. Therefore, the chosen model should allow for the modelling of different classes. While ionic-conductance models offer this capability, their implementation in SiNNs is challenging due to resource-intensive circuit requirements. On the other hand, \acrshort{lif} models are computationally inexpensive. The \acrshort{izh} model and the \acrshort{adex} model are extensions of the \acrshort{lif} that share the ability to cover multiple neuronal classes, with a slightly higher computational cost compared to the \acrshort{lif} model. They seem to be the best option but resetting the state variable in these IF-based models compromises the reproducibility of some features of neuronal activities.\\

For instance, in certain hippocampal cells, the amplitude of the spikes depends on the received stimulus \cite{Alle}. This characteristic called graded response is particularly evident \cite{Rinzel} in the Class II cells of the Hodgkin classification \cite{Hodgkin}. Notably, the assumption of spikes as stereotyped events in IF-based models prevents the faithful reproduction of this graded response.\\
Another example is the \acrfull{prc} \cite{Rinzel}, 
which delineates how the phase of a neuron in an 
oscillatory spiking state changes based on the timing 
of a pulse stimulus. Interestingly, the PRC shape for 
Class II modes in \acrshort{izh} and \acrshort{adex} 
models deviates from the standard Type II. However, 
neurons exhibiting Type II behaviour are recognized 
for their role in promoting synchronous firing 
\cite{Hansel}. Consequently, these neurons may play a 
crucial role in modelling synchronous activity in the 
brain. These pieces of evidence suggest that an IF-based
network may have a limited capacity to faithfully 
replicate certain aspects of information processing in
the nervous system.\\

Qualitative neuron models offer a lower computational cost than ioni conductance models while faithfully reproducing the dynamics of the spiking process without requiring state variable resets. The FitzHugh–Nagumo model and the Hindmarsh–Rose model \cite{Hindmarsh} represent a single neuron class and are characterised by equations containing third-degree polynomials.\\
In 2007, a \acrfull{dssn} \cite{Kohno2007,Nanami2016} model was introduced, following a qualitative approach. This model utilises quadratic terms and piecewise and step functions, enabling efficient software and digital arithmetic circuit implementation. Moreover, it is designed to represent a broad range of neuron classes \cite{NanamiNov2016}.\\

\subsection{PQN model formulation}
\label{subsection:PQN}
The PQN model \cite{Nanami} is an enhanced version of 
the DSSN model that can replicate graded responses and 
Type II PRCs. Additionally, the PQN model can emulate 
the activity of ionic-conductance models of six 
neuronal classes (\acrshort{rs}, \acrshort{fs}, 
\acrshort{ib} \cite{CONNORS}, \acrshort{lts} 
\cite{DestexheLTS}, \acrfull{eb} \cite{Izhikevich1999}, and \acrfull{eb} \cite{Ermentrout}) in response to step 
inputs of various magnitudes. Notably, its \acrshort{fpga} 
implementation consumes significantly fewer circuit 
resources than the ionic-conductance model, thanks to 
its simple equations tailored for digital arithmetic 
circuits \cite{KohnoR}. The \acrshort{pqn} model aims to offer a 
tool for constructing large-scale and biologically 
accurate \acrshort{snn} models and \acrshort{sinn}s.\\
To efficiently capture the distinct dynamics of its neuronal classes, the \acrshort{pqn} model is presented in four variations:\\

1. The \acrshort{pqn} model supports the Class II. $v$ and $n$ represent the membrane potential and recovery variable, respectively.

\begin{equation}
    \frac{dv}{dt}=\frac{\phi}{\tau}\left(f\left(v\right)-n+I_0+kI_{stim}\right)\ 
    \label{eq:PQN1a}
\end{equation}

\begin{equation}
    \frac{dn}{dt}=\frac{1}{\tau}(g\left(v\right)-n)
    \label{eq:PQN1b}
\end{equation}

In equations \ref{eq:PQN1a} and \ref{eq:PQN1b} \(f(v)\), \(g(v)\) are quadratic terms,
\(I_{0}\) is a bias constant,
\(I_{stim}\) is the stimulus input,
\(k\) is a scaling parameter,
\(\tau\), \(\phi\) are time constants of the variables.\\

2. The three-variable \acrshort{pqn} model introduces the slow variable $q$. It models the \acrshort{rs}, \acrshort{fs}, and \acrshort{eb} classes.

\begin{equation}
    \frac{dv}{dt}=\frac{\phi}{\tau}\left(f\left(v\right)-n-q+I_0+kI_{stim}\right)\ 
    \label{eq:PQN2a}
\end{equation}

\begin{equation}
    \frac{dn}{dt}=\frac{1}{\tau}(g\left(v\right)-n)    
    \label{eq:PQN2b}
\end{equation}

\begin{equation}
    \frac{dq}{dt}=\frac{\epsilon_q}{\tau}(h\left(v\right)-q)
    \label{eq:PQN2c}
\end{equation}

In equation \ref{eq:PQN2c} \(\epsilon_q\) is the time constant of the slow variable,
\(h(v)\) is a quadratic term.\\

3. The four-variable \acrshort{pqn} model adds the slow variable $u$ to replicate the \acrshort{pb} class. Here the slow subsystem consisting of $u$ and $q$ is capable of producing oscillations.

\begin{equation}
    \frac{dv}{dt}=\frac{\phi}{\tau}\left(f\left(v\right)-n-q+u+I_0+kI_{stim}\right)\ 
    \label{eq:PQN3a}
\end{equation}

\begin{equation}
    \frac{dn}{dt}=\frac{1}{\tau}(g\left(v\right)-n)
    \label{eq:PQN3b}
\end{equation}

\begin{equation}
    \frac{dq}{dt}=\frac{\epsilon_q}{\tau}(h\left(v\right)-q)
    \label{eq:PQN3c}
\end{equation}

\begin{equation}
    \frac{du}{dt}=\frac{\epsilon_u}{\tau}(v+v_0-\alpha_uu)
    \label{eq:PQN3d}
\end{equation}

In equation \ref{eq:PQN3d} \(\epsilon_u\) is the time constant of u. 
\(v_0\), \(\alpha_uu\) are model parameters.\\

4. The extended four-variable supports the \acrshort{lts} and \acrshort{ib} classes. The only difference to the previous is that the time constant of $n$ is dependent on $u$.

\begin{equation}
    \frac{dv}{dt}=\frac{\phi}{\tau}\left(f\left(v\right)-n-q+u+I_0+kI_{stim}\right)\ 
    \label{eq:PQN4a}
\end{equation}

\begin{equation}
    \frac{dn}{dt}=\frac{\eta(u)}{\tau}(g\left(v\right)-n)
    \label{eq:PQN4b}
\end{equation}

\begin{equation}
    \frac{dq}{dt}=\frac{\epsilon_q}{\tau}(h\left(v\right)-q)
    \label{eq:PQN4c}
\end{equation}

\begin{equation}
    \frac{du}{dt}=\frac{\epsilon_u}{\tau}(v+v_0-\alpha_uu)
    \label{eq:PQN4d}
\end{equation}

In equation \ref{eq:PQN4d}, \(\eta(u)\) is a heavy side step function.\\

All the parameters’ values together with their equations can be found in \cite{Nanami}.