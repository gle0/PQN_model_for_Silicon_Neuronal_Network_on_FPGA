\chapter*{Summary}
Even in the tiniest animal brains, there's a fascinating complexity at work, 
governing their bodies and actions. Despite being simple, 
these brains showcase impressive computational skills that are being unraveled. 
These processing capabilities operate with 
significantly lower energy consumption compared to computers. The study of 
the brain and its characteristics offers valuable insights into a novel and 
highly efficient form of computation that, from the biological world, can 
be exported to the artificial one.\\
By adopting this approach, the last decades saw the development of diverse 
brain-inspired systems. These technologies aim to achieve the flexibility 
and computational efficiency of biological neural processing systems, 
characterised by the implementation of neural architectures that tightly 
integrate processing and memory and the use of a spike-based encoding of the 
information.\\
Borrowing these innovative features from biology, neuromorphic engineering 
represents a transformative approach to computing. Its goal is to create 
novel, energy-efficient technologies while actively seeking to comprehend, 
through practical implementation, the remarkable capabilities of the 
brain.\\

In 2023, the Laboratory for Neuromimetics Systems at the Institute of Industrial Science, University of Tokyo, introduced the \acrfull{pqn} model. Based on a qualitative modelling approach, it outperforms traditional conductance-based models in terms of resource utilisation and power consumption while mimicking their capability of accurately modelling different neural classes. Designed for an efficient digital implementation on a \acrfull{fpga}, it represents a valid tool for constructing large-scale \acrfull{sinn}. \\
These \acrshort{sinn}s are useful for cognitive computing and computational neuroscience applications. Furthermore, their capability of mimicking the brain functioning at a cell level, utilising low power, made them a promising instrument for the development of innovative neuroprosthetic devices for the treatment of brain disorders.\\

The objective of this thesis is to expand the PQN hardware design, which presents eight different configurations, each one reproducing a distinct neuronal class, with two innovative designs. The first new configuration consists of integrating some of these designs to create a population composed of four electrophysiological classes of cortical and thalamic neurons. In the next one, cell-to-cell variability within each class is implemented by providing each neuron with a heterogeneous set of parameters.\\
Both projects share the goal of enhancing the resemblance of the \acrshort{sinn} to their biological counterpart. In particular, the modelling of two important biological features, such as diversity in the types of neuronal classes and intraclass variability, represents a step forward in the realisation of a thalamocortical microcircuit.\\

A Xilinx Artix-7 XC7A35T \acrshort{fpga}, integrated 
within a Digilent cmod A7 board, 
was used for this project.
All the source codes are openly accessible on 
GitHub\footnote{\url{https://github.com/gle0/PQN_thesis}}.