\chapter{Outlook}
\label{ch6}
The \acrshort{pqn} model ensures biological accuracy while maintaining lower power 
consumption than conductance-based models. It can model various neuron classes and 
guarantees efficient implementation through digital circuits. Its \acrshort{fpga} 
application offers high flexibility and represents an ultralow-power, large-scale, 
general-purpose silicon neuronal network platform.\\
The two designs introduced in this study both contribute to increasing the 
biological features. The first project consolidates the four principal electrophysiological 
classes of cortical and thalamic neurons into a single device, while the second one introduces 
intraclass variability. The common objective is the development of a silicon thalamocortical network.\\
To further advance the project, future endeavours should focus on reducing \acrshort{fpga} 
resource utilisation, integrating both designs synergistically, and incorporating synapses.\\

While the applications of these devices are yet to be explored, the expectations are exceedingly high. 
A multitude of event-based sensors has already been developed, such as silicon retina \cite{Mahowald}, 
silicon cochlea \cite{Lyon,Liu}, neuromorphic vestibular systems \cite{Corradi}, neuromorphic olfactory 
circuit \cite{Koickal}, and touch \cite{Bartolozzi}.\\
In addition to perception and inference, the event-based paradigm also applies 
to control \cite{Yamazaki}. Various methods have been proposed to achieve locomotion 
in a range of robots, employing a \acrfull{cpg}. It is a neural network 
wherein interconnected excitatory and inhibitory neurons generate an oscillatory, 
rhythmic output without requiring rhythmic inputs \cite{Katz}.\\

Moreover, the recent availability of extensive anatomical and physiological data 
has led to the development of data-driven conductance-based \acrshort{snn} 
models \cite{Markram,Bezaire,Ecker} with the ambitious goal of replicating 
the mammalian brain. The significant computational demand of these 
conductance-based models precludes their widespread use and the construction 
of larger-scale networks \cite{Bezaire}. In contrast, \acrshort{snn}s employing the 
\acrshort{pqn} model demonstrate a remarkable reduction in power consumption, 
making them a promising tool for these applications \cite{Nanami}.\\
Furthermore, representing a bottom-up approach with cell-level granularity 
and accurately replicating complex spiking activities of neurons, these \acrshort{sinn}s
are well-suited for artificially recreating brain regions \cite{KohnoR}. 
Consequently, it holds potential applications in biohybrid systems, including 
neuroprosthetic devices designed to treat neurological disorders and injuries \cite{Chiappalone}.\\

Brain disorders represent a leading cause of global disabilities, 
and current pharmacological treatments, along with the limited availability 
of robotic aids, fall short in providing effective solutions. This situation 
necessitates a re-evaluation of methodologies for studying human cells and addressing 
brain-related illnesses. A promising avenue in this pursuit is the exploration of 
\enquote{hybrid neuromorphic engineering} which involves integrating technological devices 
with biological neurons to devise innovative solutions \cite{Khoyratee}. Novel 
approaches to brain repair \cite{BUCCELLI} or replacement \cite{Jung,Ambroise}, 
that establish adaptive bidirectional communication through a real-time 
closed-loop architecture between biological and artificial components have 
emerged. This approach aims to overcome the limitations of current treatments 
by fostering dynamic interactions between biological and artificial systems.\\ 
Notably, the characteristics of SiNNs studied in this project align well with the 
need for a trade-off between the complexity of computational models, energy 
consumption, and real-time constraints inherent in these applications. The use 
of a \acrshort{sinn} employing the PQN model presents an exciting prospect in this context. 
Furthermore, the selection of FPGA hardware is a strategic choice to optimise the 
system's real-time performance. This not only addresses the need for efficient computation but 
also accelerates the development of an implanted bioelectronic device, offering potential 
applications in the biomedical field in the future.\\

This integrated approach holds promise for advancing our capabilities in treating 
and understanding brain-related disorders. Moreover,
exploring the interaction between artificial devices and living cells raises 
questions about the essence of intelligence and cognition. This offers the potential 
to find the boundary between the artificial and the biological life, other than the 
charming possibility of stumbling upon plausible answers to those questions.\\